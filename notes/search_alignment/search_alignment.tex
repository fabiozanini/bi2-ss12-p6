%%%%%%%%%%%%%%%%%%%%%%%%%%%%%%%%%%%%%%%%%%%%%%%%%%%%%%%%%%%%%%%%%%%%%%%%%
\documentclass[12pt,a4paper,notitlepage,onecolumn]{article}
%%%%%%%%%%%%%%%%%%%%%%%%%%%%%%%%%%%%%%%%%%%%%%%%%%%%%%%%%%%%%%%%%%%%%%%%%
\newcommand{\Author}{Fabio Zanini}
\newcommand{\Title}{Search and alignment}
\newcommand{\Keywords}{{bioinformatics}, {project}}
%%%%%%%%%%%%%%%%%%%%%%%%%%%%%%%%%%%%%%%%%%%%%%%%%%%%%%%%%%%%%%%%%%%%%%%%%
\usepackage[english]{babel}
\usepackage[utf8x]{inputenc}
\usepackage[top=2cm]{geometry}
\usepackage{amsmath,amsfonts,amssymb,eucal,eurosym}
\usepackage{color}
\usepackage{graphicx}
\usepackage[font=small, format=hang, labelfont={sf,bf}, figurename=Fig.]{caption}
\usepackage{cite}
%\usepackage{epigraph}
%\setlength{\epigraphwidth}{.55\textwidth}
%\setlength{\epigraphrule}{0pt}
%\usepackage{multirow}
%\usepackage[version=3]{mhchem}
%\usepackage{sagetex}
\usepackage[	colorlinks,linkcolor=red,citecolor=red]{hyperref}
\hypersetup{	pdfauthor={\Author}, pdftitle={\Title}, pdfkeywords={\Keywords}	}
%%%%%%%%%%%%%%%%%%%%%%%%%%%%%%%%%%%%%%%%%%%%%%%%%%%%%%%%%%%%%%%%%%%%%%%%%
%\graphicspath{{./figures/}}
%%%%%%%%%%%%%%%%%%%%%%%%%%%%%%%%%%%%%%%%%%%%%%%%%%%%%%%%%%%%%%%%%%%%%%%%%
%\DeclareMathOperator\de{d\!}
%\newcommand{\comment}[1]{\textit{\textcolor{red}{#1}}}
%%%%%%%%%%%%%%%%%%%%%%%%%%%%%%%%%%%%%%%%%%%%%%%%%%%%%%%%%%%%%%%%%%%%%%%%%
\title{\Title}
\author{\Author}
\date{\today}
%%%%%%%%%%%%%%%%%%%%%%%%%%%%%%%%%%%%%%%%%%%%%%%%%%%%%%%%%%%%%%%%%%%%%%%%%
\begin{document}
\maketitle

The given sequence is taken from the PDB entry 1YJE (\url{http://www.rcsb.org/pdb/explore/explore.do?structureId=1YJE}). The paper that describes it is Ref.~\cite{flaig_structural_2005}. I checked this by diff the text files, exactly the same label even.

CSBlasting it against the non-redundant PDB using Lupas's toolkit reports quite a few sequences with a reasonably low e-value (number of entries that are reported by chance, according to Ref.~\cite{karlin_methods_1990}). CSBlast also outputs an MSA of the sequences, so I could check directly how closely related they were to our ``unknown'' sequence:
\begin{enumerate}
 \item the first sequence is ours, found back from the PDB;
 \item the next ones, up to $e\sim 0.01$, have reasonable chunks in common with our sequence;
 \item the remaining ones have only a domain in common, either at the N-terminus or in the middle.
\end{enumerate}

I therefore selected only the first ones ($e < 0.01$) and aligned them using MUSCLE with 6 iterations. It took 3 seconds, meaning that they were already aligned \texttt{:-)}. Now we should go for modeller.

\bibliographystyle{plain}
\bibliography{search_alignment}




\end{document}

